% % % % % % % % % % % % % % % % % % % % % % % % % % % % % % % % % % % % % % % %
% LaTeX4EI Example for Cheat Sheets
%
% @encode: 	UTF-8, tabwidth = 4, newline = LF
% @author:	LaTeX4EI
% % % % % % % % % % % % % % % % % % % % % % % % % % % % % % % % % % % % % % % %


% ======================================================================
% Document Settings
% ======================================================================

% possible options: color/nocolor, english/german, threecolumn
% default: color, english
\documentclass[english]{latex4ei/latex4ei_sheet}
\usepackage{dsfont}
\usepackage{textcomp}
\usepackage{bbm}
\usepackage{breqn}
\usepackage{mathtools}
\usepackage{polynom}
% set document information
\title{Model\\Predictive\\Control}


% DOCUMENT_BEGIN ===============================================================
\begin{document}

\maketitle	% requires ./img/Logo.pdf

\section{Basics}
\begin{sectionbox}
\textbf{Compact set}: Subset of Euclidean space being closed (\textit{i.e., containing all its limit points}) and bounded (\textit{i.e., having all its points lie within some fixed distance of each other}).\\
\textbf{Matrix Inversion}: $A^{-1}=\frac{1}{\det{A}}\cdot\textrm{adj}(A)$
$$\begin{bmatrix}a & b \\ c & d\end{bmatrix}^{-1} = \frac{1}{ad-bc} \begin{bmatrix}d & -b \\ -c & a\end{bmatrix}$$
\textbf{Eigenvalues}: $\det(A-\lambda I)=0$\\
\textbf{Eigenvectors}: $(A-\lambda I)\cdot v=0$\\
\textbf{Rule of Sarrus}: $A_{3\times3}=\begin{bmatrix}a_{11} & a_{12} & a_{13} \\ a_{21} & a_{22} & a_{23} \\ a_{31} & a_{32} & a_{33} \end{bmatrix}$
\begin{multline*}
    \det(A_{3\times3})=a_{11}a_{22}a_{33}+a_{12}a_{23}a_{31}+a_{13}a_{21}a_{32}\\-a_{31}a_{22}a_{13}-a_{32}a_{23}a_{11}-a_{33}a_{21}a_{12}
\end{multline*}
For system equations $y_i=f_i(x)$, you can find $x$ with \textbf{Least Squares}: \\ $$x=\arg \min_{x} \left[\sum\limits_{i=1}^{n}\left(y_i-f_i(x)\right)^2\right]$$
($\rightarrow$ \textit{Calculate derivative, set to zero and find corresponding value for} $x$)

\end{sectionbox}

\section{Introduction}
\begin{sectionbox}

\subsection{System Class}
Time-invariant discrete-time dynamical control system \\
$x(k+1)=f(x(k), u(k));\quad k=0,1,2,...\quad x(0)=x_0$\\
with state $x(k)\in\mathbb{R}^n$ and control input $u(k)\in\mathbb{R}^m$.\\

Notation:\\
${\underline{u}^{N}\coloneqq\{u(0), u(1), \ldots u(N-1)\}} \\ {\underline{x}_{u}^{N}\left(x_{0}\right)\coloneqq\left\{x_{0}, x_{\underline{u}}\left(1, x_{0}\right), \ldots x_{\underline{u}}\left(N, x_{0}\right)\right\}}$ \\ 
or if context is clear for brevity: \\ ${\underline{x}_{\underline{u}}^{N}\left(x_{0}\right)\coloneqq\left\{x_{0}, x(1), \ldots x(N)\right\}}$\\

\subsection{Cost Function}
Infinite Horizon:
\[J_{\infty}\left(x_{0}, \underline{u}^{\infty}\right)=\sum_{k=0}^{\infty} l\left(x_{\underline{u}}\left(k, x_{0}\right), u(k)\right)\]
Finite Horizon:
\[J_{N}\left(x_{0}, \underline{u}^{N}\right)=\sum_{k=0}^{N-1} l\left(x_{\underline{u}}\left(k, x_{0}\right), u(k)\right)+J_{f}\left(x_{\underline{u}}\left(N, x_{0}\right)\right)\]
with stage cost $l(x,u)$ and target cost $J_f(x)$.\\

\subsection{Constraints}
Input constraints: $u(k) \in \mathbb{U}$ \\ 
State constraints: $x(k) \in \mathbb{X}, k=1,2,3, \ldots ;\; x(N) \in \mathbb{X}_{f}$ \\ \\
Admissible Controls:  $\mathcal{U}_{N}\left(x_{0}\right)\coloneqq\left\{\underline{u} |\left(x_{0}, \underline{u}\right) \in \mathbb{Z}\right\}$ \\ Feasible Initial Values: $\mathcal{X}_{N}\coloneqq\left\{x_{0} \in \mathbb{X} | \mathcal{U}_{N}\left(x_{0}\right) \neq \emptyset\right\}$\\
with 
\begin{multline*}
\mathbb{Z}_{N}\coloneqq\Big\{\left(x_{0}, \underline{u}\right) | u(k) \in \mathbb{U}, x_{\underline{u}}\left(k, x_{0}\right) \in \mathbb{X}, \\ k=0,1, \ldots, N-1 ; x_{\underline{u}}\left(N, x_{0}\right) \in \mathbb{X}_{f}\Big\}
\end{multline*}

\end{sectionbox}
\begin{sectionbox}

\subsection{Optimization Problem}
$$
\mathbb{P}_{N}\left(x_{0}\right): J_{N}^{*}\left(x_{0}\right)=\min _{\underline{u}}\left\{J_{N}\left(x_{0}, \underline{u}\right) | \underline{u} \in \mathcal{U}_{N}\left(x_{0}\right)\right\}
$$
\\
Assumption 1:\\
$f,\,l,\,J_f$ are continuous with $f(0,0)=0,\; l(0,0)=0,\; J_f(0)=0$.\\
\\
Assumption 2:\\
$\mathbb{X}$ is closed, $\mathbb{X}_f$ and $\mathbb{U}$ are compact and all sets contain the origin.\\
\\
Under \textit{Assumption 1} and \textit{Assumption 2}, the optimization problem $\mathbb{P}_N(x_0)$ has a solution for all $x_0\in\mathcal{X}_N$. \\($\rightarrow$ Theorem of Weierstrass: Sets are sequentially compact \textit{and} every bounded sequence of complex numbers contains at least one convergent subsequence.)\\

\subsection{Controller}
$\kappa_{N}\left(x_{0}\right)=u^{*}\left(0, x_{0}\right)$ \\ 
with optimal control input $u^{*}\left(0, x_{0}\right)$ from solution of $\mathbb{P}_{N}\left(x_{0}\right)$, \\$\underline{u}^{*}=\left\{u^{*}\left(0, x_{0}\right), u^{*}\left(1, x_{0}\right), \ldots u^{*}\left(N-1, x_{0}\right)\right\}$.\\

\subsection{Basic time-invariant MPC algorithm}
System: $x^{+}=f(x, u)$\\ 
Cost: $J(x, \underline{u})=\sum\limits_{k=0}^{N-1} l(x(k), u(k))+J_{f}(x(N))$ \\
Constraints: $x(k) \in \mathbb{X}, u(k) \in \mathbb{U} \text { for all } k \in \mathbb{N}_{0} \text { and } x(N) \in \mathbb{X}_{f}$ \\
where $N$ is the prediction horizon.
\begin{itemize}
    \item Measure $x$, determine $\mathcal{U}_N(x)$
    \item Solve $\mathbb{P}_N(x)$ and obtain $\underline{u}^*(x)$
    \item Control with $\kappa_N(x)$ such that $x^+=f(x,\kappa_N(x))$
    \item Repeat for $x\coloneqq x^+$
\end{itemize}\vspace{0.2cm}

\subsection{Constrained Optimization in a nutshell}
Cost function: $\min F(z)$\\
Equality constraints: $g(z)=0$\\
Inequality constraints: $h(z)\leq 0$\\
\\
\textbf{a)} unconstrained\\
If $z^*$ is minimum, then $\nabla F(z^*)=0$.\\
\textbf{b)} equality constrained\\
Lagrange function $L=F(z)+\lambda^{\top} g(z)$ with Lagrange multiplier $\lambda$. \\ 
If $z^{*}$ is minimum, then $\nabla_{z} L\left(z^{*}, \lambda^{*}\right)=0$ and $\nabla_{\lambda} L\left(z^{*}, \lambda^{*}\right)=0$. \\
\textbf{c)} inequality constrained\\
Lagrange function $L=F(z)+\mu^{\top} h(z)$ with KKT multiplier $\mu$. \\
If $z^{*}$ is minimum then $\nabla_{z} L\left(z^{*}, \mu^{*}\right)=0$ and $\mu^{*} \geq 0, h\left(z^{*}\right) \leq 0$ and $\mu_{i}^{*} h_{i}\left(z^{*}\right)=0$ for all $i$.\\
\\
Optimum for cost function at horizon $N$ for $J_N=0$. If only one constraint is active, use \textbf{b)} to look at constrained edge of control input and derive the remaining optimal control inputs.

\end{sectionbox}
\vspace{5cm}
\section{Dynamic Programming}
\begin{sectionbox}

\subsection{Problem Statement}
Time-invariant discrete-time dynamical control system\\
$x(k+1)=f(x(k), u(k)) ; \quad k=0,1,2, \ldots \quad x(0)=x_{0}$ \\ \\
Cost: $V\left(x_{0}, \underline{u}\right)=\sum\limits_{k=0}^{N-1} l(x(k), u(k))+V_{f}(x(N))$ \\
State constraints: $x(k) \in \mathbb{X},\; k=0,1, \ldots, N-1,\; x(N) \in \mathbb{X}_{f}$\\
Input constraints: $u(k) \in \mathbb{U}, k=1,2, \ldots, N-1$ \\ \\
Find $\underline{u}=\{u(0), u(1), \ldots, u(N-1)\}$ where $u(k)=\mu_{k}(x(k))$ are control laws.\\

\subsection{Notation}
Cost-to-go: $V_{i}\left(x, \underline{u}^{i}\right)=\sum\limits_{k=i}^{N-1} l(x(k), u(k))+V_{f}(x(N))$ \\
with $\underline{u}^{i}=\{u(i), u(i+1), \ldots, u(N-1)\}$ \\ \\
Optimal cost-to-go: $V_{i}^{*}(x)=\min _{\underline{u}^{i} \in \Upsilon_{i}(x)} V_{i}\left(x, \underline{u}^{i}\right)$ \\
with
\begin{multline*} \Upsilon_{i}(x)\coloneqq\\\left\{\underline{u}^{i} | \text { for initial state } x(i)=x: u(k) \in \mathbb{U}, k=i, \ldots, N-1;\right.\\\left.x(k) \in \mathbb{X}, k=i+1, \ldots, N-1 ; x(N) \in \mathbb{X}_{f}\right\} 
\end{multline*}
and $\Xi_{i}\coloneqq\left\{x \in \mathbb{X} | \Upsilon_{i}(x) \neq \emptyset\right\}$ \\
\\
Recursive construction of feasible sets $\Xi_{i}$ from behind:\\ 
$\Xi_{N}=\mathbb{X}_{f}$ \\ 
$\Xi_{i}=\left\{x(i) \in \mathbb{X} | x(i+1) \in \Xi_{i+1} \text { with } u(i) \in \mathbb{U}\right\},\\
i=N-1, N-2, \ldots 0$\\

\subsection{Bellman Recursion}
Recursive calculation of optimal-cost $V_i^*$, starting with the terminal cost: \\
$V_{N}^{*}(x(N))=V_f(x(N))$ \; $\rightarrow$ \; $V_{N-1}^{*}=\min\{V^{*}_N+\ldots\}$ \; $\rightarrow$ \;\ldots\\
In general:
\begin{multline*}
V_{i}^{*}(x(i))=\min _{u(i)}\Big\{l(x(i), u(i))+V_{i+1}^{*}\left(f(x(i), u(i))\right) \;| \\u(i) \in \mathbb{U}, x(i) \in \mathbb{X}, f(x(i), u(i)) \in \Xi_{i+1}\Big\},\\
i=N-1,N-2,\ldots,0
\end{multline*}
Use system $f(x(i),u(i))$ to substitute $x(i+1)$ in $V^{*}_{i+1}(x(i+1))$ and solve $\frac{\partial V_{i}}{\partial u(i)}=0$ to get optimal control input $u^{*}(i)=ax(i)$.\\ Insert $u^{*}(i)$ into $V^{*}_{i}(x(i))$ to get optimal cost-to-go.

\end{sectionbox}

\section{Stability}
\begin{sectionbox}

\subsection{Stability Concepts}
System class: $x^{+}=\Tilde{f}(x)$\\
Equilibrium point: $x_{eq}=\Tilde{f}(x_{eq})$\\
Open loop: $x^{+}=f(x)$ \; $\longleftrightarrow$ \; Closed loop: $x^{+}=f(x,u)$\\
\\
Assumption:\\
If $\Tilde{f}$ is not continuous, it is at least locally bounded.\\
\\
Definition of stability in the sense of Lyapunov:\\
The equilibrium point $x_{\mathrm{eq}}=0$ of $x^{+}=\hat{f}(x)$ is locally stable, if for all $\varepsilon>0$ there exists a $\delta>0$ such that for all $\|x(0)\| \leq \delta(\varepsilon)$, it holds that $\|x(k)\| \leq \varepsilon$ for all $k>0$.
\begin{itemize}
    \item Locally asymptotically stable, if in addition $\lim _{k \rightarrow \infty}\|x(k)\|=0$ for $x(0)$ close to the origin.
    \item Globally asymptotically stable, if in addition $\lim _{k \rightarrow \infty}\|x(k)\|=0$ for all $x(0) \in \mathbb{R}^{n}$.
    \item Asymptotically stable in $\mathcal{X}$, if in addition $\lim _{k \rightarrow \infty}\|x(k)\|=0$ for all $x(0) \in \mathcal{X}$, where $\mathcal{X}$ is positive.
\end{itemize}

\end{sectionbox}
\begin{sectionbox}

Definition of positive invariance:\\
$\mathcal{X}$ is positive invariant for $x^{+}=\Tilde{f}(x)$, if $\Tilde{f}(x)\in\mathcal{X}$ for all $x\in\mathcal{X}$.\\
(\textit{Once a system trajectory enters set} $\mathcal{X}$, \textit{it will never leave it again.})
\\ \\
Definition of comparison functions:
\begin{itemize}
    \item A function $\alpha$ is a class $\mathcal{K}$ function, if it is continuous and strictly increasing with $\alpha(0)=0$.
    \item A function $\alpha$ is a class $\mathcal{K}_{\infty}$ function, if it a class $\mathcal{K}$ function and in addition unbounded.
    \item A function $\alpha$ is a class $\mathcal{PD}$ (\textit{positive definite}) function, if it is continuous with $\alpha(0)=0$ and $\alpha(x)>0$ for all $x\neq0$.
\end{itemize}
\vspace{0.1cm}\\ 

Lyapunov's direct method:\\
A function $V\,:\,\mathbb{R}^{n}\:\rightarrow\:\mathbb{R}$ is a global Lyapunov function for the equilibrium of $x^{+}=\Tilde{f}(x)$, if $\alpha_1,\alpha_2\in\mathcal{K}_{\infty},\,\alpha_3\in\mathcal{PD}$ exist, such that for all $x\in\mathbb{R}^n$ condition \textbf{(1)} and \textbf{(2)} holds.\\
If $V$ is a global Lyapunov function for the equilibrium of $x^{+}=\Tilde{f}(x)$, then the equilibrium is globally asymptotically stable.\\
\\
Lyapunov's direct method (\textbf{constrained}):\\
A function $V\,:\,\mathcal{X}\:\rightarrow\:\mathbb{R}$ with $\mathcal{X}$ invariant is a Lyapunov function for the equilibrium of $x^{+}=\Tilde{f}(x)$ on $\mathcal{X}$, if $\alpha_1,\alpha_2\in\mathcal{K}_{\infty},\,\alpha_3\in\mathcal{PD}$ exist, such that for all $x\in\mathcal{X}$ condition \textbf{(1)} and \textbf{(2)} holds.\\
If $V$ is a Lyapunov function for the equilibrium of $x^{+}=\Tilde{f}(x)$ on $\mathcal{X}$, then the equilibrium is asymptotically stable on $\mathcal{X}$.\\

\begin{emphbox}
    (1) Bounded: $\alpha_{1}(\|x\|) \leq V(x) \leq \alpha_{2}(\|x\|)$ \\ 
    (2) Descent: $V(f(x))-V(x) \leq-\alpha_{3}(\|x\|)$\\
\end{emphbox}

\subsection{Assumptions for Stability}
Assumption 3:\\
$l(x, u)>\alpha_{l}(\|x\|) \quad \forall x \in \mathcal{X}_{N}, \forall u \in \mathbb{U}$ \\ 
$J_{f}(x) \leq \alpha_{f}(\|x\|) \quad \forall x \in \mathbb{X}_{f}$ where $\alpha_{l}, \alpha_{f}$ are class $\mathcal{K}_{\infty}$ functions.\\
\\
Assumption 4:\\
$J_f$ is a Control Lyapunov Function (CLF), that means $J_f(0)=0,\ J_f>0$ and there exists $u\in\mathbb{U}$ such that $J_f(f(x,u))-J_f(x)\leq-l(x,u)\ \forall x\in\mathbb{X}_f$.\\
If system is CLF, it is feedback stabilizable.\\
\\
Assumption 5:\\
$\mathbb{X}_f$ is control invariant, that means, if $x\in\mathbb{X}_f$ then there exists $u\in\mathbb{U}$ such that $f(x,u)\in\mathbb{X}_f$.\\ 

\subsection{Stability of MPC}
Under Assumptions 1 to 5, the equilibrium $x_{eq}$ is asymptotically stable in $\mathcal{X}_N$ for $x^{+}=f(x,\kappa_{N}(x))$.\\
\textit{Proof}: Choose Lyapunov function $V_N(x)=J_N(x,\underline{u}^{*})$ and show property (1) and (2) of Lyapunov direct method.\\

\subsection{Recursive Feasibility}
Definition: MPC is said to be recursively feasible, if one can assure that there is a solution to $\mathbb{P}_N(x^{+})$ having a solution of $\mathbb{P}_N(x)$.\\
\\
Recursive Feasibility:\\
If $\mathbb{X}_f$ is control invariant, then
\begin{itemize}
    \item $\mathcal{X}_{j-1}\subseteq\mathcal{X}_{j},\;j=1,\ldots,N$
    \item $\mathcal{X}_{j-1}$ is control invariant, $j=1,\ldots,N$
    \item MPC is recursively feasible
\end{itemize}
\vspace{0.2cm}
To find feasible initial values $\mathcal{X}_1$ for an additional target constraint $\mathcal{X}_0=\mathbb{X}_f=\{0\}$, check if system dynamics satisfy target constraint with $x^{+}=f(x,u)=0$.\\
\\
To determine $\mathcal{X}_1$ given an control invariant target set $\mathcal{X}_0=\mathbb{X}_f$, put in system dynamics $x^{+}$ for $x$.

\end{sectionbox}

\newpage
\section{MPC for Linear Systems}
\begin{sectionbox}

\subsection{System Class for Linear MPC}
System class: $x^{+}=A x+B u$ \\
Cost: $J_{N}(x, \underline{u})=\frac{1}{2} \sum\limits_{k=0}^{N-1}\|x(k)\|_{Q}^{2}+\|u(k)\|_{R}^{2}+\frac{1}{2}\|x(N)\|_{P_{f}}^{2}$ \\
Constraints: $x(k) \in \mathbb{X}, u(k) \in \mathbb{U}, x(N) \in \mathbb{X}_{f}$, where $\mathbb{X}$, $\mathbb{U}$ and $\mathbb{X}_{f}$ are convex polytopes\\

\subsection{LQ Control (no constraints, non-receding finite horizon)}
System Class: $x^{+}=A x+B u$ \\
Cost: $$V_{0}\left(x_{0}, u\right)=\frac{1}{2} \sum_{k=1}^{N-1}\underbrace{\|x(k)\|_{Q}^{2}}_{\substack{\text{penalize} \\ \text{bad performance}}}+\underbrace{\|u(k)\|_{R}^{2}}_{\substack{\text{penalize} \\ \text{actuator effort}}}+\frac{1}{2}\|x(N)\|_{P_{f}}^{2}$$ \\
Control law: $u(k)=K(k) x(k)$ \\
\\
while for $k=0, \ldots, N-1$, with $P(N)=P_{f}$
$$K(k)=-\left(B^{\top} P(k+1) B+R\right)^{-1} B^{\top} P(k+1) A$$ and
\begin{multline*}
    \quad\ P(k)=A^{\top} P(k+1) A+Q-A^{\top} P(k+1) B\\\left(B^{\top} P(k+1) B+R\right)^{-1} B^{\top} P(k+1) A
\end{multline*}
Recursion for the Riccatti Matrix:
$$P(k)=A^{\top} P(k+1) A+Q+\boldsymbol{K}(k)^{\top}B^{\top} P(k+1)A$$
In addition:\\ 
$V_{0}^{*}\left(x_{0}\right)=\frac{1}{2} x_{0}^{\top} P(0) x_{0}$\\

\subsection{LQ Control (no constraints, infinite horizon)}
System Class: $x^{+}=A x+B u$ \\
Cost: $V(x, \underline{u})=\frac{1}{2} \sum\limits_{k=0}^{\infty}\|x(k)\|_{Q}^{2}+\|u(k)\|_{R}^{2}$ \\
\\
Control Law: $u(k)=K_{\infty} x(k)$, while \\ $K_{\infty}=-\left(B^{\top} P_{\infty} B+R\right)^{-1} B^{\top} P_{\infty} A$ and Riccati Equation:\\ 
${P_{\infty}=Q+K_{\infty}^{\top} R K_{\infty}+\left(A+B K_{\infty}\right)^{\top} P_{\infty}\left(A+B K_{\infty}\right)}$\\
\\
Stationary Riccatti Matrix:
$$P_{\infty}=A^{\top} P_{\infty} A+Q+\boldsymbol{K}_{\infty}^{\top}B^{\top} P_{\infty}A,\qquad P_{\infty}\geq 0$$

\subsection{MPC (constrained, receding horizon)}
Stability of equilibrium $x_{eq}=0$ under MPC, if\\
unconstrained: $P_{f}=P_{\infty}$\\
constrained:\\
1.) $P_{f}=P_{\infty}$ \\
2.) constraint admissibility: $\mathbb{X}_{f} \subseteq\{x \in \mathbb{X} | K x \in \mathbb{U}\}$ \\
3.) positive invariance: $x \in \mathbb{X}_{f} \Rightarrow x^{+}=\left(A+B K_{\infty}\right) x \in \mathbb{X}_{f}$\\

\subsection{Underlying Optimization Problem (QP)}
Using previews $x(k)=A^{k}x_{0}+A^{k-1}Bu(0)+\ldots+Bu(k-1)$\\
for all $k$ in horizon in cost function 
$$J_{N}\left(x_{0}, \underline{u}\right)=\frac{1}{2} \sum\limits_{k=0}^{N-1}\|x(k)\|_{Q}^{2}+\|u(k)\|_{R}^{2}+\frac{1}{2}\|x(N)\|_{P_{f}}^{2}$$
allows to transform the cost into $$J_N(x,\underline{u})=\frac{1}{2}\underline{u}^{\top}H(x_0)\underline{u}+c(x_{0})^{\top}\underline{u}+d(x_{0})$$
with $\underline{u}\in\mathcal{U}_{N}(x_{0})$
polytopes QP problem (= quadratic cost with linear constraints) allows for efficient numerics.

\end{sectionbox}


\section{Generalized Predictive Control (GPC)}
\begin{sectionbox}

\subsection{System Class}
System class:\\ 
$$A\left(z^{-1}\right) y(t)=B\left(z^{-1}\right) z^{-d} u(t-1)+C\left(z^{-1}\right) \frac{e(t)}{\Delta}$$
\begin{itemize}
    \item Denominator: $A\left(z^{-1}\right)=1+a_{1} z^{-1}+a_{2} z^{-2}+\ldots+a_{n} z^{-n}$
    \item Denumerator: $B\left(z^{-1}\right)=1+b_{1} z^{-1}+\ldots+b_{m} z^{-m}$, $m<n$
    \item Shift operator: $z^{-k} y(t)=y(t-k)$
    \item Dead time: $z^{-d}$, in the following $d=0$
    \item $e(t)$: white noise with zero mean
    \item $\Delta=1-z^{-1}$ (for $u(t) \leftrightarrow$ for $u(t-1)$: $\Delta=1$)
    \item $C\left(z^{-1}\right)$ for colored noise, in the following $C\left(z^{-1}\right)=1$
\end{itemize}
Cost: 
$$J=\sum_{j=1}^{N} \delta(j)(\underbrace{\hat{y}(t+j | t)}_{\substack{\text{predicted} \\ \text{output}}}-\underbrace{w(t+j)}_{\substack{\text{future ref.} \\ \text{trajectory}}})^{2}+\sum_{j=1}^{M} \lambda(j)(\Delta u(t+j-1))
$$
\begin{itemize}
    \item Horizons: $N$ is prediction horizon, $M$ is control horizon, in the following $M=N$
    \item Weighting $\delta(j),\ \lambda(j)$
    \item Control input $\Delta u = u(t)-u(t-1)$
    \item Prediction from time $t$ to $t+j$:
\end{itemize}
$$\hat{y}(t+j | t)=G_{j}(z^{-1})\Delta u(t+j-1)+F_{j}(z^{-1})y(t)
$$

\subsection{Diophantine Equation}
$\rightarrow$ Fewer equations than unknown variables\\
$1=E_{j} z^{-1} \tilde{A}\left(z^{-1}\right)+z^{-j} F_{j}\left(z^{-1}\right)$ with $\tilde{A}=\Delta A$ \\ $E_{j}\left(z^{-1}\right)$: polynomial of degree $j-1$ \\ $F_{j}\left(z^{-1}\right)$: polynomial of degree of $\tilde{A}$
\\ \\

Prediction:\\ $\underline{y}=\underline{G u}+\underline{p}$ \quad($\underline{p}$ often denoted as $f$)\\
where the choice of $\underline{y}$ and $\underline{u}$ as:
$$
\begin{array}{l}{\underline{y}=[\hat{y}(t+1 | t), \hat{y}(t+2 | t), \ldots, \hat{y}(t+N | t)]^{\top}} \\ 
\underline{u}=[\Delta u(t), \Delta u(t+1), \ldots, \Delta u(t+N-1)]^{\top}
\end{array}
$$
defines $G$ and $\underline{p}$:
$$G=\left[\begin{array}{cccc}{g_{0}} & {0} & {\dots} & {0} \\ {g_{1}} & {g_{0}} & {\dots} & {0} \\ {\vdots} & {\vdots} & {\vdots} & {\vdots} \\ {g_{N-1}} & {g_{N-2}} & {\dots} & {g_{0}}\end{array}\right]$$
$$\underline{p}=\underline{F}\left(z^{-1}\right) y(t)+\underline{G}^{\prime}\left(z^{-1}\right) \Delta u(t-1)$$
with\\
$G_{j}\left(z^{-1}\right)=B\left(z^{-1}\right) E_{j}\left(z^{-1}\right)$\\
$\underline{G}^{\prime}\left(z^{-1}\right)=&\left[\begin{array}{c}{\left(G_{1}\left(z^{-1}\right)-g_{0}\right) z} \\ {\left(G_{2}\left(z^{-1}\right)-g_{0}-g_{1} z^{-1}\right) z^{2}}\end{array}\right]$\\
$\underline{F}\left(z^{-1}\right)=\left[F_{1}\left(z^{-1}\right), F_{2}\left(z^{-1}\right), \ldots, F_{N}\left(z^{-1}\right)\right]^{\top}$\\
and $g_{j}$ $j$-th coefficients of polynomial $G_{j}$.

\end{sectionbox}
\begin{sectionbox}
\textbf{Excursion:}
Polynomials $E_{j}\left(z^{-1}\right)$ and $F_{j}\left(z^{-1}\right)$ can be obtained by dividing $1$ by $\tilde{A}(z^{-1})=\Delta A(z^{-1})$ until the remainder can be factorized as $z^{-j}F_{j}(z^{-1})$, e.g.:\\
\( \begin{aligned} (1 \qquad \ \qquad):(\overbrace{1-2z^{-1}+2z^{-2}-z^{-3}}^{\Delta A(z^{-1})})=\overbrace{\overbrace{1}^{E_1}+2z^{-1}}^{E_2}+\ \ldots \end{aligned}\\
\begin{aligned}
\frac{1-2 z^{-1}+2 z^{-2}-z^{-3}}{\qquad 2 z^{-1}-2 z^{-2}+z^{-3}} &  \frac{}{\ \ \qquad \qquad \Rightarrow z^{-1} F_{1}\left(z^{-1}\right)}
\end{aligned}\\
\begin{aligned}
\frac{\qquad 2 z^{-1}-4 z^{-2}+4 z^{-3}-2 z^{-4}}{\qquad \qquad \qquad 2z^{-2}- 3z^{-3}+2 z^{-4}} & \frac{}{\Rightarrow z^{-2} F_{2}\left(z^{-1}\right)} & \ldots \end{aligned} \)
\\
For $N=2$:\\
$E_1(z^{-1})=1$, \quad $E_2(z^{-1})=1+2z^{-1}$\\
\\
$z^{-1}F_1(z^{-1})=2z^{-1}-2z^{-2}+z^{-3}$\\ \textcolor{white}{n}$\rightarrow F_1(z^{-1})=2\qquad -2z^{-1}+z^{-2}$\\
\\
$z^{-2}F_2(z^{-1})=2z^{-2}-3z^{-3}+2z^{-4}$\\ \textcolor{white}{n}$\rightarrow F_2(z^{-1})=2\qquad -3z^{-1}+2z^{-2}$\\

\subsection{QP Problem}
Cost: $$J=u\left(G^{T} Q G+R\right) \underline{u}+2(\underline{p}-\underline{w})^{T} Q G \underline{u}+(\underline{p}-\underline{w})^{T} Q(\underline{p}-\underline{w})$$ is minimized by control sequence of future controls \\
$\underline{u}=-\left(G^{T} Q G+R\right)^{-1} G^{T} Q(\underline{p}-w)$\\
where only $\underline{u}_{1}=\Delta u(t)$ is applied as control.

\end{sectionbox}


\section{Numerics}
\begin{sectionbox}

\subsection{Nonlinear Programming (NP)}
Cost function: $F(z)$\\
Equality constraints: $g(z)=0$\\
Inequality constraints: $h(z)\leq 0$\\
\\
Necessary conditions for a minimum:\\
If $z^{*}$ is feasible minimum, then $\nabla_{z} L\left(z^{*}, \lambda^{*}, \mu^{*}\right)=0$, $\nabla_{\lambda} L\left(z^{*}, \lambda^{*}, \mu^{*}\right)=0$ and $\mu^{*} \geq 0$, $h\left(z^{*}\right) \leq 0$ and $\mu_{i}^{*} h_{i}\left(z^{*}\right)=0$ for all $i$ with Lagrange function $L=F(z)+\lambda^{\top} g(z)+\mu^{\top} h(z)$ with multipliers $\lambda$ and $\mu$.\\
Active Set: $\mathcal{A}=\left\{j | \mu_{j}>0\right\}$\\

\subsection{Unconstrained Minimization: Newton's Method}
Find minimum $z^{*}$ numerically as follows
\begin{itemize}
    \item Initialize: Guess $z^{(0)}$ close to $z^{*}$ for $k=0,1,2,\ldots$
    \item Update (calculate tangent and corresponding root value):\\
    $z^{(k+1)}=z^{(k)}+\alpha^{(k)} d^{(k)}$ where\\ $d^{(k)}=-\left(\nabla^{2} F\left(z^{(k)}\right)\right)^{-1} \nabla F\left(z^{(k)}\right)$ (search direction)\\
    and $\alpha^{(k)}=\textrm{argmin}\ F(z^{(k)}+\alpha d^{(k)})$ (line search)
    \item Stop if $\|\nabla F(z^{(k)})\|<\epsilon_{tol}$
\end{itemize}\vspace{0.2cm}

\end{sectionbox}
\textbf{Notes:}
\begin{sectionbox}

\subsection{Constrained Minimization: Quadratic Programming}
Applies for QP problem class:\\
Quadratic cost function: $F(z)=\frac{1}{2}z^{\top}Hz+c^{\top}z$\\
Linear equality constraints: $g(z)=Ez+e=0$\\
Liner inequality constraints: $h(z)=Iz+i\leq 0$\\
\\
Find minimum $z^{*}$ numerically as follows
\begin{itemize}
    \item Initialize: Guess initial active set $\mathcal{A}_0$ for $k=0,1,2,\ldots$
    \item Update optimization variables:
\end{itemize}
$$\left(\begin{array}{ccc}{H} & {E^{\top}} & {\left(I^{(k)}\right)^{\top}} \\ {E} & {0} & {0} \\ {I^{(k)}} & {0} & {0}\end{array}\right)\left(\begin{array}{l}{z^{(k+1)}} \\ {\lambda^{(k+1)}} \\ {\mu^{(k+1)}}\end{array}\right)=\left(\begin{array}{c}{-c} \\ {-e} \\ {-i^{(k)}}\end{array}\right)$$
where $I^{(k)}z+i^{(k)}\leq 0$ are inequality constraints from active set $\mathcal{A}_k$ that are treated as equality constraints.
\begin{itemize}
    \item Update active set:\\
    When for $j \in \mathcal{A}_{k}: \mu_{j}^{(k+1)}<0$, delete const. $j$ from active set\\
    When for $j \notin \mathcal{A}_{k} I_{j} z^{(k+1)}+i_{j} \geq 0$, add constraint $j$ to active set and thus get $\mathcal{A}_{k+1}$
    \item Stop if $\mathcal{A}_{k}$ is not active anymore.
\end{itemize}\vspace{0.2cm}

\subsection{Constrained Minimization: Sequential QP}
Combine Newton's Methods (linearization of NP problem to obtain QP problem) and QP method (choice of feasible active set of linearized problem).\\

\subsection{Constrained Minimization: Interior Point}
Cost function: $F(z)$\\
Equality constraints: $g(z)=0$\\
Former inequality constraints, now equality constraints: $h(z)+s=0$\\
New inequality constraints: $s\geq 0$\\
\\
Lagrange Function $L=F(z)+y^{\top}g(z)+w^{\top}(h(z)+s)-\mu^{\top}s$\\
Necessary Conditions that are related to inequality constraints:\\
$s \geq 0 $, $\mu \geq 0$ and $s^{\top}\mu=0$\\
\\
Relax complementary conditions $s^{\top}\mu=0$ by introducing barrier parameter $\epsilon$ to $s^{\top}\mu=\epsilon$ and solve with $\epsilon \rightarrow 0$.
\end{sectionbox}


\section{Robust MPC}
\begin{sectionbox}

\subsection{Types of Uncertainties}
Parametric uncertainty, modeling errors, measurement noise, etc.\\
E.g. additive disturbance: $x^{+}=f(x,u)+w$ where $w$ is disturbance and $w\in\mathbb{W}$, where $\mathbb{W}$ is bounded.\\
\subsection{Robust Stability}
Under bounded small disturbances, does the state reach a small vicinity of the origin.\\
Compare formal definitions RGAS or Practical Stability in the literature.\\
\\
Nominal Robust Stability only if Lyapunov Function is continuous (e.g. for linear MPC)!\\

\subsection{Tube base MPC for Linear Systems}
Nominal model: $z^{+}=Az+Bu$\\
Disturbed (real) model: $x^{+}=Ax+Bu+w$\\
Use tube $S(k)$, $k=0,1,2,\ldots$ to check constraint admissibility: $S(k)$ is a set such that all possible (apply all possible disturbances!) trajectories $x(k)$ of the disturbed system fulfill $x(k)\in\{z(k) \oplus S(k)\}$\\
MPC feedback control: $u^{*}=v^{*}+K(x-z)$, where $K$ is found offline. Appropiate choice of $K$ reduces the size of the tube.\\
Optimize cost to find $v^{*}$ such that $x^{*}$ is constraint admissible by constraint tightening using the tube.

\end{sectionbox}


% DOCUMENT_END =================================================================
\end{document}
